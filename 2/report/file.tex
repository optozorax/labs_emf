\pgfplotstableset{
	begin table={\rowcolors{2}{gray!25}{white}\begin{tabular}},
	end table=\end{tabular},
}

\mytitlepage{прикладной математики}{1}{Уравнения математической физики}{Решение эллиптических краевых задач методом конечных разностей}{ПМ-63}{Шепрут И.И.}{10}{Патрушев И.И.}{2019}

\section{Цель работы}

Разработать программу решения эллиптической краевой задачи методом конечных разностей. Протестировать программу и численно оценить порядок аппроксимации.

\section{Задание}

\noindent\begin{easylist}
\ListProperties(Hang=true, Margin2=12pt, Style2**=$\bullet$ , Hide2=2, Hide1=0)
& Построить прямоугольную сетку в области в соответствии с заданием. Допускается использовать фиктивные узлы для сохранения регулярной структуры.
& Выполнить конечноразностную аппроксимацию исходного уравнения в соответствии с заданием. Получить формулы для вычисления компонент матрицы A и вектора правой части b.
& Реализовать программу решения двумерной эллиптической задачи методом конечных разностей с учетом следующих требований:
&& язык программирования С++ или Фортран;
&& предусмотреть возможность задания неравномерных сеток по пространству, учет краевых условий в соответствии с заданием;
&& матрицу хранить в диагональном формате, для решения СЛАУ использовать метод блочной релаксации [2, стр. 886], [3].
& Протестировать разработанные программы на полиномах соответствующей степени.
& Провести исследования порядка аппроксимации реализованного методов для различных задач с неполиномиальными решениями. Сравнить полученный порядок аппроксимации с теоретическим.
\end{easylist}

\textbf{Вариант 10:} Область имеет Ш-образную форму. Предусмотреть учет первых и третьих краевых условий.

\section{Анализ задачи}

Необходимо решить задачу:

$$ \Delta u = f(x, y) $$

Первые краевые условия записываются в виде: $ u(x, y)|_\Gamma = g_1(x, y) $, где $ g_1(x, y) $ --- известная функция. 

Третьи краевые условия записываются в виде: $ u'(x, y) + A\cdot u(x, y) |_\Gamma = g_3(x, y) $ где $ g_3(x, y) $ --- известная функция. 

На первом этапе решения задачи нужно построить сетку. Матрица формируется одним проходом по всем узлам, для регулярных узлов заполняется согласно пятиточечному шаблону, для прочих – в соответствии с краевыми условиями.

\section{Таблицы и графики}

Параметры:

\noindent\begin{easylist}
\ListProperties(Hang1=true, Margin2=12pt, Style1**=$\bullet$ , Hide1=1)
& Сетка имеет Ш-образную форму.
& Область сетки: $[0, 1]\times[0, 1]$.
& Количество элементов: 20 по обеим осям.
& Сетка равномерная.
& Точностью решения является норма $L_2$ разности векторов истинных значений и значений, вычисленных с помощью конечно-разностной аппроксимации.
& СЛАУ решается при помощи метода Гаусса-Зейделя с точностью $10^{-12}$, максимальным числом итераций $3000$ и параметром релаксации $1.4$.
\end{easylist}

\subsection{Точность для разных функций}

\begin{center}
\noindent\pgfplotstabletypeset[
	columns={a,$1$,$u$,$u^2$,$u^2+1$,$u^3$,$u^4$,$e^u$,sinu},
	columns/a/.style={string type, column name={\backslashbox{$u(x, t)$}{$\lambda(u)$}}},
	columns/$1$/.style={string type},
	columns/$u$/.style={string type},
	columns/$u^2$/.style={string type},
	columns/$u^2+1$/.style={string type},
	columns/$u^3$/.style={string type},
	columns/$u^4$/.style={string type},
	columns/$e^u$/.style={string type},
	columns/sinu/.style={string type, column type/.add={}{|},},
	every head row/.style={before row=\hline,after row=\hline\hline}, 
	every last row/.style={after row=\hline},
	column type/.add={|}{},
	col sep=tab,
]{first.txt}
\end{center}

\subsection{Зависимость точности от размера сетки}

Параметры остаются прежними, с небольшими изменениями:

% \noindent\begin{easylist}
% \ListProperties(Hang1=true, Margin2=12pt, Style1**=$\bullet$ , Hide1=1)
% & Количество элементов по обеим осям является переменной величиной.
% & Для неполиномиальной функции $e^{x^3+yx^2}$.
% \end{easylist}

\newcommand{\graphgrid}[1]{
\noindent\begin{tikzpicture}
\begin{semilogyaxis}[xlabel=t,ylabel=Точность решения,width=\textwidth, height=6cm]
\addplot[red, no markers] table [skip first n=1, y=r1, x=t]{#1.txt};
\addplot[blue, no markers] table [skip first n=1, y=r2, x=t]{#1.txt};
\addplot[black, no markers] table [skip first n=1, y=ru, x=t]{#1.txt};
\legend{$x^t$,$x^\frac{1}{t}$,$x$}
\end{semilogyaxis}
\end{tikzpicture}

\noindent\begin{tikzpicture}
\begin{semilogyaxis}[xlabel=t,ylabel=Точность решения,width=\textwidth, height=6cm]
\addplot[green, no markers] table [skip first n=1, y=r3, x=t]{#1.txt};
\addplot[orange, no markers] table [skip first n=1, y=r4, x=t]{#1.txt};
\addplot[black, no markers] table [skip first n=1, y=ru, x=t]{#1.txt};
\legend{$\frac{t^x-1}{t-1}$,$\frac{\frac{1}{t}^x-1}{\frac{1}{t}-1}$,$x$}
\end{semilogyaxis}
\end{tikzpicture}

\noindent\begin{tikzpicture}
\begin{semilogyaxis}[xlabel=t,ylabel=Число итераций,width=\textwidth, height=6cm]
\addplot[red, no markers] table [skip first n=1, y=i1, x=t]{#1.txt};
\addplot[blue, no markers] table [skip first n=1, y=i2, x=t]{#1.txt};
\addplot[black, no markers] table [skip first n=1, y=iu, x=t]{#1.txt};
\legend{$x^t$,$x^\frac{1}{t}$,$x$}
\end{semilogyaxis}
\end{tikzpicture}

\noindent\begin{tikzpicture}
\begin{semilogyaxis}[xlabel=t,ylabel=Число итераций,width=\textwidth, height=6cm]
\addplot[green, no markers] table [skip first n=1, y=i3, x=t]{#1.txt};
\addplot[orange, no markers] table [skip first n=1, y=i4, x=t]{#1.txt};
\addplot[black, no markers] table [skip first n=1, y=iu, x=t]{#1.txt};
\legend{$\frac{t^x-1}{t-1}$,$\frac{\frac{1}{t}^x-1}{\frac{1}{t}-1}$,$x$}
\end{semilogyaxis}
\end{tikzpicture}
}

\newcommand{\graphgridtime}[1]{
\noindent\begin{tikzpicture}
\begin{semilogyaxis}[xlabel=t,ylabel=Точность решения,width=\textwidth, height=6cm]
\addplot[red, no markers] table [skip first n=1, y=r1, x=t]{#1.txt};
\addplot[blue, no markers] table [skip first n=1, y=r2, x=t]{#1.txt};
\addplot[green, no markers] table [skip first n=1, y=r3, x=t]{#1.txt};
\addplot[orange, no markers] table [skip first n=1, y=r4, x=t]{#1.txt};
\addplot[black, no markers] table [skip first n=1, y=ru, x=t]{#1.txt};
\legend{$x^t$,$x^\frac{1}{t}$,$\frac{t^x-1}{t-1}$,$\frac{\frac{1}{t}^x-1}{\frac{1}{t}-1}$,$x$}
\end{semilogyaxis}
\end{tikzpicture}

\noindent\begin{tikzpicture}
\begin{semilogyaxis}[xlabel=t,ylabel=Число итераций,width=\textwidth, height=6cm]
\addplot[red, no markers] table [skip first n=1, y=i1, x=t]{#1.txt};
\addplot[blue, no markers] table [skip first n=1, y=i2, x=t]{#1.txt};
\addplot[green, no markers] table [skip first n=1, y=i3, x=t]{#1.txt};
\addplot[orange, no markers] table [skip first n=1, y=i4, x=t]{#1.txt};
\addplot[black, no markers] table [skip first n=1, y=iu, x=t]{#1.txt};
\legend{$x^t$,$x^\frac{1}{t}$,$\frac{t^x-1}{t-1}$,$\frac{\frac{1}{t}^x-1}{\frac{1}{t}-1}$,$x$}
\end{semilogyaxis}
\end{tikzpicture}
}

\graphgrid{x4_space}

\graphgrid{expx_space}

\graphgridtime{t3_time}

\graphgridtime{expt_time}


% \subsection{Зависимость точности от параметра разрядки неравномерной сетки}

% Параметры остаются прежними, с небольшими изменениями:

% \noindent\begin{easylist}
% \ListProperties(Hang1=true, Margin2=12pt, Style1**=$\bullet$ , Hide1=1)
% & Область сетки: $[0, 1]\times[0, 1]$ для красных графиков, и $[1, 2]\times[2, 3]$ для синих графиков.
% & Сетка неравномерная.
% & Коэффициент разрядки неравномерной сетки является переменной величиной.
% \end{easylist}

% Для коэффициента разрядки 1 сетка получается равномерной.

% \newcommand{\inputgraph}[1]{
% \noindent\begin{tikzpicture}
% \begin{semilogyaxis}[xlabel=Коэффициент разрядки неравномерной сетки,ylabel=Точность решения,width=\textwidth, height=6cm]
% \addplot[red, no markers] table [y=norm, x=c]{0_0_non_uniform_grid_#1.txt};
% \addplot[blue, no markers] table [y=norm, x=c]{1_2_non_uniform_grid_#1.txt};
% \end{semilogyaxis}
% \end{tikzpicture}	
% }

% \subsubsection{$2x+y$}
% \inputgraph{0}

% \subsubsection{$3x^2+y^2$}
% \inputgraph{1}

% \subsubsection{$x^3+xy^2+y^3$}
% \inputgraph{2}

% \subsubsection{$x^4+y^4$}
% \inputgraph{3}

% \subsubsection{$x^5+y^5+2xy$}
% \inputgraph{4}

% \subsubsection{$e^{x+y}$}
% \inputgraph{5}

% \subsubsection{$e^{x^2+y^2}$}
% \inputgraph{6}

% \subsubsection{$e^{x^3+yx^2}$}
% \inputgraph{7}

% \subsubsection{$\sin x+\cos y$}
% \inputgraph{8}

% \subsubsection{$\sqrt{x^2+y^2}$}
% \inputgraph{9}

% \subsubsection{$x^{1.2}+y^{1.5}$}
% \inputgraph{10}

% \section{Выводы}

% \noindent\begin{easylist}
% \ListProperties(Hang1=true, Margin2=12pt, Style1**=$\bullet$ , Hide2=1, Hide1=1)
% & Порядок аппроксимации метода конечных разностей с первыми краевыми условиями равен 3, с третьими краевыми условиями равен 1.
% & На неравномерных сетках с первыми краевыми условиями можно получить решение, стремящееся к точному, изменяя коэффициент разрядки.
% & Согласно графику зависимости точности решения от размера сетки, можно вычислить, что порядок сходимости равен 1.
% & Для неполиномиальных функций смещение области расчета может изменять точность решения.
% \end{easylist}

% \section{Код программы}

% Для решения СЛАУ и хранения матрицы в диагональном формате был использован код из 2 лабораторной работы по Численным Методам.

% \subsection{Основной код}

% \mycodeinput{c++}{../main.cpp}{main.cpp}

% \subsection{Код из другой лабораторной работы}

% \subsubsection{Заголовочные файлы}

% \mycodeinput{c++}{../nm2/2/diagonal.h}{diagonal.h}
% \mycodeinput{c++}{../nm2/1/vector.h}{vector.h}
% \mycodeinput{c++}{../nm2/1/common.h}{common.h}
% \mycodeinput{c++}{../nm2/1/matrix.h}{matrix.h}

% \subsubsection{Файлы исходного кода}

% \mycodeinput{c++}{../nm2/2/diagonal.cpp}{diagonal.cpp}
% \mycodeinput{c++}{../nm2/1/vector.cpp}{vector.cpp}
% \mycodeinput{c++}{../nm2/1/common.cpp}{common.cpp}
% \mycodeinput{c++}{../nm2/1/matrix.cpp}{matrix.cpp}