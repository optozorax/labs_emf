\pgfplotstableset{
	begin table={\rowcolors{2}{gray!25}{white}\begin{tabular}},
	end table=\end{tabular},
}

\mytitlepage{прикладной математики}{2}{Уравнения математической физики}{Решение нелинейных начально-краевых задач}{ПМ-63}{Шепрут И.И.}{5}{Патрушев И.И.}{2019}

\section{Цель работы}

Разработать программу решения нелинейной одномерной краевой задачи методом конечных элементов. Провести сравнение метода простой итерации и метода Ньютона для решения данной задачи.

\section{Задание}

\begin{easylist}
\ListProperties(Hang=true, Margin2=12pt, Style2**=$\bullet$ , Hide2=2, Hide1=0)
& Выполнить конечноэлементную аппроксимацию исходного уравнения в соответствии с заданием. Получить формулы для вычисления компонент матрицы A и вектора правой части b для метода простой итерации.
& Реализовать программу решения нелинейной задачи методом простой итерации с учетом следующих требований:
&& язык программирования С++ или Фортран;
&& предусмотреть возможность задания неравномерных сеток по пространству и по времени, разрывность параметров уравнения по подобластям, учет краевых условий;
&&  матрицу хранить в ленточном формате, для решения СЛАУ использовать метод LU -разложения;
&& предусмотреть возможность использования параметра релаксации.
& Протестировать разработанную программу.
& Провести исследования реализованных методов на различных зависимостях коэффициента от решения (или производной решения) в соответствии с заданием. На одних и тех же задачах сравнить по количеству итераци метод простой итерации. Исследовать скорость сходимости от параметра релаксации.
\end{easylist}

\textbf{Вариант 5:} уравнение $-\operatorname{div}\left(\lambda(u)\operatorname{grad} u\right) + \sigma \frac{\partial u}{\partial t} = f$. Базисные функции  - линейные.

\section{Исследования}

Далее под точностью решения будет подразумеваться $L_2$ норма между вектором $q$, полученным в ходе решения, на последнем моменте времени, и между реальным значением узлов, которые мы знаем, задавая функцию $u$. 

В исследованиях на порядок сходимости эта норма будет ещё делиться на число элементов, для нахождения среднего отклонения от идеального решения.

Хотя, наверное, можно было бы считать общую площадь между истинной функцией и полученной аппроксимацией при помощи интеграла разности двух этих функций. Тогда появляется вопрос: как нам считать точность решения при неравномерной сетке? Очевидно, она может плохо аппроксимировать те части уравнения, на которых она сильно разрежена. Хотя, насколько сильно нас интересуют эти части? И насколько сильно точность по всей функции будет отличаться от точности по элементам сетки. Это всё интересные вопросы для потенциальных исследований, выходящие за рамки данной работы.

Далее под числом итераций будет подразумеваться сумма числа итераций по нелинейности по всем временным точкам. Например, если у нас имеется сетка по времени из 10 элементов, и на каждом шаге было совершено 10 итераций методом простой итерации, то итоговое число итераций будет равно 100.

Так же далее для всех исследований будут использоваться следующие параметры, изменение этих параметров будет оговариваться отдельно.

\begin{easylist}
\ListProperties(Hang1=true, Margin2=12pt, Style1**=$\bullet$ , Hide1=1)
& $sigma = 1$.
& $\varepsilon = 0.001$.
& $\mathrm{iters}_{max} = 500$.
& Функция правой части высчитывается автоматически.
& Сетка по пространству равномерная: $(1, 1.1, \dots, 1.9, 2)$. 
& Сетка по времени равномерная: $(0, 0.1, \dots, 0.9, 1)$.
& Начальное приближение для функций $u$, линейных по $t$ --- $(1, 1, \dots)$.
& Для функций $u$, нелинейных по $t$ начальное приближение в момент $t=0$ --- $(u(1, 0), u(1.1, 0), $ $ \dots, u(1.9, 0), u(2, 0))$, то есть истинное решение в этот момент времени.
\end{easylist}

\subsection{Точность для разных функций}

Здесь показана точность решения и количество итераций в зависимости от функций $u(x, t)$ и $\lambda(u)$. $\varepsilon = 10^{-7}$.

\begin{center}
\noindent\pgfplotstabletypeset[
	columns={a,$1$,$u$,$u^2$,$u^2+1$,$u^3$,$u^4$,$e^u$,sinu},
	columns/a/.style={string type, column name={\backslashbox{$u(x, t)$}{$\lambda(u)$}}},
	columns/$1$/.style={string type},
	columns/$u$/.style={string type},
	columns/$u^2$/.style={string type},
	columns/$u^2+1$/.style={string type},
	columns/$u^3$/.style={string type},
	columns/$u^4$/.style={string type},
	columns/$e^u$/.style={string type},
	columns/sinu/.style={string type, column name={$\sin u$}, column type/.add={}{|},},
	every head row/.style={before row=\hline,after row=\hline\hline}, 
	every last row/.style={after row=\hline},
	column type/.add={|}{},
	col sep=tab,
]{first.txt}
\end{center}

\textbf{Вывод:} порядок аппроксимации сильно зависит от нелинейности. Например, при отсутствии нелинейности порядок аппроксимации нулевой, а при нелинейности вплоть до 3 степени, порядок аппроксимации равен $\infty$, так как довольно точно аппроксимируется функция $e^x$. При $\lambda(u) = u^4$ порядок аппроксимации вообще равен 1.

\textbf{Вывод:} порядок аппроксимации по времени равен 1.

\subsection{Зависимость точности от нелинейной сетки}

\subsubsection{Функции нелинейной сетки}

В ходе выполнения лабораторной работы были обнаружены функции, позволяющие легко задавать неравномерную сетку, сгущающуюся к одному из концов.

Если у нас задано начало --- $a$ и конец сетки --- $b$, а количество элементов $n$, тогда сетку можно задать следующим образом:

$$ x_i = a + m\left(\frac{i}{n}\right) \cdot (b-a), i=\overline{0, n} $$

где $m(x)$ --- некоторая функция, задающая неравномерную сетку. При этом $x$ обязан принадлежать области $[0, 1]$, а функция $m$ возвращать значения из той же области, и при этом быть монотонной на этом участке. Тогда гарантируется условие монотонности сетки, то есть что при $j \leqslant i\, \Rightarrow\, x_j \leqslant x_i$. 

\textit{Пример:} при $m(x) = x$, сетка становится равномерной.

Найденные функции зависят от некоторого параметра t:

\begin{center}\begin{tabular}{cc}
$ m_{1, t}(x) = x^t $ & $ m_{2, t}(x) = x^\frac{1}{t} $ \\
$ m_{3, t}(x) = \frac{t^x-1}{t-1} $ & $ m_{4, t}(x) = \frac{\frac{1}{t^x}}{\frac{1}{t}-1} $
\end{tabular}\end{center}

Что интересно, эти функции вырождаются в $x$ при $t=1$, а при $t=0$, они вырождаются в сетку, полностью находящуюся на одном из концов: 1, 3 фукнции стремятся к концу $b$; а функции 2, 4 стремятся к концу $a$. 1 и 2 функции симметричны друг другу, как 3 и 4.

Таким образом, можно исследовать различные неравномерные сетки на итоговую точность и число итераций, изменяя параметр от $0$ до $1$.

\subsubsection{Описание исследований}

Параметры остаются прежними, с небольшими изменениями:

\begin{easylist}
\ListProperties(Hang1=true, Margin2=12pt, Style1**=$\bullet$ , Hide1=1)
& $\mathrm{iters}_{max} = 100$.
& Сетка по пространству неравномерная, если исследование происходит по сетке пространству, и равномерная, если исследование происходит по сетке времени.
\end{easylist}

Исследуется скорость и качество сходимости в зависимости от параметра неравномерной сетки. Так же некоторые графики разделены для того, чтобы можно было что-то различить на них.

Графики точности и количества итераций обрезались сверху, так как данные о плохой точности и большом числе итераций нас не интересуют, так же эти данные мешают анализировать хорошие зависимости.

\newcommand{\graphgrid}[1]{
\noindent\begin{tikzpicture}
\begin{semilogyaxis}[xlabel=t,ylabel=Точность решения,width=\textwidth, height=6cm]
\addplot[red, no markers] table [skip first n=1, y=r1, x=t]{#1.txt};
\addplot[blue, no markers] table [skip first n=1, y=r2, x=t]{#1.txt};
\addplot[black, no markers] table [skip first n=1, y=ru, x=t]{#1.txt};
\legend{$m_{1, t}$,$m_{2, t}$,$x$}
\end{semilogyaxis}
\end{tikzpicture}

\noindent\begin{tikzpicture}
\begin{semilogyaxis}[xlabel=t,ylabel=Точность решения,width=\textwidth, height=6cm]
\addplot[green, no markers] table [skip first n=1, y=r3, x=t]{#1.txt};
\addplot[orange, no markers] table [skip first n=1, y=r4, x=t]{#1.txt};
\addplot[black, no markers] table [skip first n=1, y=ru, x=t]{#1.txt};
\legend{$m_{3, t}$,$m_{4, t}$,$x$}
\end{semilogyaxis}
\end{tikzpicture}

\noindent\begin{tikzpicture}
\begin{semilogyaxis}[xlabel=t,ylabel=Число итераций,width=\textwidth, height=6cm]
\addplot[red, no markers] table [skip first n=1, y=i1, x=t]{#1.txt};
\addplot[blue, no markers] table [skip first n=1, y=i2, x=t]{#1.txt};
\addplot[black, no markers] table [skip first n=1, y=iu, x=t]{#1.txt};
\legend{$m_{1, t}$,$m_{2, t}$,$x$}
\end{semilogyaxis}
\end{tikzpicture}

\noindent\begin{tikzpicture}
\begin{semilogyaxis}[xlabel=t,ylabel=Число итераций,width=\textwidth, height=6cm]
\addplot[green, no markers] table [skip first n=1, y=i3, x=t]{#1.txt};
\addplot[orange, no markers] table [skip first n=1, y=i4, x=t]{#1.txt};
\addplot[black, no markers] table [skip first n=1, y=iu, x=t]{#1.txt};
\legend{$m_{3, t}$,$m_{4, t}$,$x$}
\end{semilogyaxis}
\end{tikzpicture}
}

\newcommand{\graphgridtime}[1]{
\noindent\begin{tikzpicture}
\begin{semilogyaxis}[xlabel=t,ylabel=Точность решения,width=\textwidth, height=6cm]
\addplot[red, no markers] table [skip first n=1, y=r1, x=t]{#1.txt};
\addplot[blue, no markers] table [skip first n=1, y=r2, x=t]{#1.txt};
\addplot[green, no markers] table [skip first n=1, y=r3, x=t]{#1.txt};
\addplot[orange, no markers] table [skip first n=1, y=r4, x=t]{#1.txt};
\addplot[black, no markers] table [skip first n=1, y=ru, x=t]{#1.txt};
\legend{$m_{1, t}$,$m_{2, t}$,$m_{3, t}$,$m_{4, t}$,$x$}
\end{semilogyaxis}
\end{tikzpicture}

\noindent\begin{tikzpicture}
\begin{semilogyaxis}[xlabel=t,ylabel=Число итераций,width=\textwidth, height=6cm]
\addplot[red, no markers] table [skip first n=1, y=i1, x=t]{#1.txt};
\addplot[blue, no markers] table [skip first n=1, y=i2, x=t]{#1.txt};
\addplot[green, no markers] table [skip first n=1, y=i3, x=t]{#1.txt};
\addplot[orange, no markers] table [skip first n=1, y=i4, x=t]{#1.txt};
\addplot[black, no markers] table [skip first n=1, y=iu, x=t]{#1.txt};
\legend{$m_{1, t}$,$m_{2, t}$,$m_{3, t}$,$m_{4, t}$,$x$}
\end{semilogyaxis}
\end{tikzpicture}
}

\subsubsection{Сетка по пространству}

В данных исследованиях неравномерность применяется к сетке по пространству.

\myparagraph{u = x4 + t}

Исследуется функция $u(x, t) = x^4 + t$, при $\lambda(u) = u^2$.

\graphgrid{x4_space}

\textbf{Вывод:} у обеих функций имеются точки, где точность больше, чем при равномерной сетке --- это точки в окрестности 0.9. Гладкость графика точности в зависимости от параметра позволяет найти эти точки, например, при помощи метода Ньютона, спускаясь из $t=1$. Но к сожалению точность увеличивается на ничтожно малую величину, поэтому использование неравномерных сеток на основе данных функций может быть нецелесообразно. Так же в этих точках немного увеличивается число итераций.

\myparagraph{u = exp(x) + t}

Исследуется функция $u(x, t) = e^x + t$, при $\lambda(u) = u^2$.

\graphgrid{expx_space}

\textbf{Вывод:} предыдущий вывод подтверждается ещё одним примером, разве что теперь точность увеличилась не на ничтожную величину, а на половину порядка.

\subsubsection{Сетка по времени}

В данных исследованиях неравномерность применяется к сетке по времени.

\myparagraph{u = exp(x) + t3}

Исследуется функция $u(x, t) = e^x + t^3$, при $\lambda(u) = u$.

\graphgridtime{t3_time}

\textbf{Вывод:} точность лучше, чем у равномерной сетки, показали 1 и 3 функция, которые сгущаются к последнему моменту времени $t=1$. Наверное  благодаря этому точность и увеличивается. Причем видно, что как и у 1, так и у 3 функции есть явно выраженные минимумы, но к сожалению функция зависимости точности от параметра не идеально гладкая, а с небольшими ступеньками, что не позволяет настолько просто подобрать нужный параметр при помощи градиентных методов. Хотя, к сожалению, точность увеличивается на совершенно ничтожную величину. Так же любая неравномерная сетка требует меньше итераций, чем равномерная, что может быть аргументом в пользу их использования.

\myparagraph{u = exp(x) + exp(t)}

Исследуется функция $u(x, t) = e^x + e^t$, при $\lambda(u) = u$.

\graphgridtime{expt_time}

\textbf{Вывод:} предыдущий вывод подтвердился. Так же можно заметить, что при некоторых параметрах сетки число итераций становится вообще минимально возможным --- 10.

\subsection{Точность в зависимости от размера сетки}

В данном пункте определяется \textit{порядок сходимости}. \textbf{Порядок сходимости} лучше всего описать на примере. Например, у нас есть сетка, и точность на ней равна 10, мы уменьшаем эту сетку вдвое, и точность становится 5. Тогда порядок сходимости равен 1, если же итоговая точность станет 2.5, то порядок сходимости будет 2. Порядок сходимости --- это степень того, насколько сильно увеличивается точность при уменьшении сетки на какой-то параметр. Порядок сходимости можно определить из степени, которая стоит у $x$, в функции, хорошо аппроксимирующей исходные данные.

\subsubsection{Сетка по пространству}

Исследуется на функции $u(x, t) = e^x+e^t$, при $\lambda(u) = u$.

\noindent\begin{tikzpicture}
\begin{semilogyaxis}[xlabel=Количество элементов сетки,ylabel=Точность решения,width=\textwidth, height=6cm]
\addplot[black, no markers, line width=1pt] table [skip first n=1, y=residual, x=size]{expx_expt_size_space.txt};
\addplot[orange, dashed, line width=1.5pt, smooth, domain=3:202] {0.0255*x^(-0.5)};
\addplot[red, smooth, domain=3:202] {0.0255*5^(0.5)*x^(-1)};
\legend{Точность решения,$0.0255\cdot x^{-0.5}$,$0.0255\cdot \sqrt{5}\cdot x^{-1}$}
\end{semilogyaxis}
\end{tikzpicture}

\textbf{Вывод:} порядок сходимости по пространственной сетке равен $0.5$.

Так же имеется число итераций в зависимости от размера сетки:

\noindent\begin{tikzpicture}
\begin{axis}[xlabel=Количество элементов сетки,ylabel=Число итераций,width=\textwidth, height=6cm]
\addplot[black, no markers] table [skip first n=1, y=iterations, x=size]{expx_expt_size_space.txt};
\end{axis}
\end{tikzpicture}

\textbf{Вывод:} при увеличении числа элементов пространственной сетки, число итераций падает, хоть и незначительно.

\subsubsection{Сетка по времени}

Исследуется на функции $u(x, t) = e^x+e^t$, при $\lambda(u) = u$.

\noindent\begin{tikzpicture}
\begin{semilogyaxis}[xlabel=Количество элементов сетки,ylabel=Точность решения,width=\textwidth, height=6cm]
\addplot[black, no markers, line width=1pt] table [skip first n=1, y=residual, x=size]{expx_expt_size_time.txt};
\addplot[red, dashed, line width=1.5pt, smooth, domain=3:202] {0.08*x^(-1)};
\addplot[blue, smooth, domain=3:202] {0.08*5*x^(-2)};
\legend{Точность решения,$0.08\cdot x^{-1}$,$0.08\cdot 5\cdot x^{-2}$}
\end{semilogyaxis}
\end{tikzpicture}

\textbf{Вывод:} порядок сходимости по временной сетке равен $1$.

Так же имеется число итераций в зависимости от размера сетки:

\noindent\begin{tikzpicture}
\begin{axis}[xlabel=Количество элементов сетки,ylabel=Число итераций,width=\textwidth, height=6cm]
\addplot[black, no markers, line width=1pt] table [skip first n=1, y=iterations, x=size]{expx_expt_size_time.txt};
\addplot[red, dashed, smooth, domain=3:202] {x};
\legend{Решение,$y=x$}
\end{axis}
\end{tikzpicture}

\textbf{Вывод:} начиная с определенного числа элементов временной сетки, число итераций равно числу элементов временной сетки. Что означает, что в методе простой итерации происходит минимальное возможное число итераций --- 1.

Интересно было бы сравнить данные результаты с результатами, когда использовалась интегральная норма, а не норма между векторами. Может оказаться, что увеличение размера сетки по пространству может оказаться выгодней увеличения сетки по времени, несмотря на то, что порядок сходимости у временной сетки лучше.

% \section{Выводы}

% \noindent\begin{easylist}
% \ListProperties(Hang=true, Margin2=12pt, Style1**=$\bullet$ , Hide2=1, Hide1=1)
% & Порядок аппроксимации метода конечных элементов с линейными элементами с первыми краевыми условиями равен ? при нелинейности $\lambda = 1$, и ? при нелинейности $\lambda = u$.
% & TODO...
% & TODO...
% & TODO...
% & TODO...
% \end{easylist}

\section{Код программы}

\mycodeinput{c++}{../main.cpp}{main.cpp}