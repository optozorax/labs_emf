\pgfplotstableset{
	begin table={\rowcolors{2}{gray!25}{white}\begin{tabular}},
	end table=\end{tabular},
}

% Собственные обозначения для скобок
\newcommand{\roubr}[1]{\left(#1\right)}  % round  brackets ()
\newcommand{\sqbr}[1]{\left[#1\right]}   % sqaure brackets [] 
\newcommand{\cubr}[1]{\left\{#1\right\}} % curtly brackets {}

% Обозначение для описания выводов.
\newcommand{\conclusion}[0]{\noindent\textbf{Вывод: }}

% Жирный текст в математике
\newcommand{\mb}[1]{\mathbf{#1}}

\newcommand{\myd}[1]{\operatorname{d}\!#1}

% Дивергенция и градиент
\newcommand{\mydiv}[0]{\operatorname{div}}
\newcommand{\mygrad}[0]{\operatorname{grad}}
\newcommand{\mypartial}[2]{\frac{\partial #1}{\partial #2}}
\newcommand{\mypartialpow}[3]{\frac{\partial^{#3} #1}{\partial #2^{#3}}}
\newcommand{\myintegral}[2]{\int\limits_{#1}^{#2}}

\newcommand{\scalemath}[2]{\scalebox{#1}{\mbox{\ensuremath{\displaystyle #2}}}}

\begin{center}
\hfill \break
\Large{Министерство науки и высшего образования Российской Федерации}\\
\hfill \break
\large{Федеральное государственное бюджетное образовательное учреждение высшего образования}\\ 
\normalsize{\textbf{«НОВОСИБИРСКИЙ ГОСУДАРСТВЕННЫЙ ТЕХНИЧЕСКИЙ УНИВЕРСИТЕТ»}}\\
\hfill \break
\includegraphics{nstu_logo.eps} \\
\hfill \break
\large{Кафедра прикладной математики}\\
\hfill \break
\large{Метод конечных элементов}\\
\hfill \break
\hfill \break
\Large{\textbf{Пояснительная записка к курсовому проекту}}\\
\hfill \break
\hfill \break 
\normalsize{\begin{tabular}{cllp{1.5cm}p{1.5cm}}
\multirow{5}{*}[0.75cm]{\includegraphics[scale=0.5]{fami_logo.eps}}
& \textbf{Факультет:} & ПМИ & & \\[1.25ex]
& \textbf{Группа:} & ПМ-61 & & \\[1.25ex]
& \textbf{Студент:} & Шепрут И.И. & & \\[1.25ex]
& \textbf{Преподаватель:} & Персова М.Г. & & \\[1.25ex]
&  &  & & \\[1.25ex]
\end{tabular}} \\
\hfill \break
\hfill \break
\hfill \break
\hfill \break
\hfill \break
\large{Новосибирск\\2020}
\end{center}
\thispagestyle{empty}
\newpage 
\setcounter{page}{1}

\section{Задание}

Реализовать МКЭ для двумерной задачи для гиперболического уравнения в декартовой системе координат. Базисные функции --- биквадратичные. Схема Кранка-Николсона.

\section{Теория}

Решаемое уравнение в общем виде:

$$ -\mydiv\roubr{\lambda\mygrad u} + \gamma u + \sigma \mypartial{u}{t} + \chi \mypartialpow{u}{t}{2} = f $$

Решаемое уравнение в декартовой двумерной системе координат:

$$ -\mypartial{}{x}\roubr{\lambda\mypartial{u}{x}} - \mypartial{}{y}\roubr{\lambda\mypartial{u}{y}} + \gamma u + \sigma \mypartial{u}{t} + \chi \mypartialpow{u}{t}{2} = f $$

Первые краевые условия:

$$ u|_{S} = u_s $$

Формулы для билинейных базисных функций прямоугольных элементов:

$$ X_1(x) = 
1 + \frac{3 x_{p}}{h_{x}} + \frac{2 x^{2}}{h_{x}^{2}} - \frac{x \left(3 h_{x} + 4 x_{p}\right)}{h_{x}^{2}} + \frac{2 x_{p}^{2}}{h_{x}^{2}}
$$

$$ X_2(x) = 
- \frac{4 x^{2}}{h_{x}^{2}} + \frac{4 x \left(h_{x} + 2 x_{p}\right)}{h_{x}^{2}} - \frac{4 x_{p} \left(h_{x} + x_{p}\right)}{h_{x}^{2}}
$$

$$ X_3(x) = 
\frac{2 x^{2}}{h_{x}^{2}} - \frac{x \left(h_{x} + 4 x_{p}\right)}{h_{x}^{2}} + \frac{x_{p} \left(h_{x} + 2 x_{p}\right)}{h_{x}^{2}}
$$

Формулы $Y_{1, 2, 3}(y)$ выглядят аналогично $X_{1, 2, 3}(x)$, только с заменой $x \rightarrow y$, $h_x \rightarrow h_y$.

Конечный элемент для биквадратичных базисов представляется так:

\begin{center}
\begin{tikzpicture};
	\tikzstyle{ann} = [fill=white,font=\footnotesize,inner sep=1pt];
	\coordinate (O) at ($ (0, 0) $);
	\coordinate [label=right:$x$] (X) at ($ (5, 0) $);
	\coordinate [label=right:$y$] (Y) at ($ (0, 4.5) $);
	\coordinate (MX) at ($ (-0.5, 0) $);
	\coordinate (MY) at ($ (0, -0.5) $);

	\coordinate (A) at ($ (1,1) $);
	\coordinate (B) at ($ (2.5,1) $);
	\coordinate (C) at ($ (4,1) $);
	\coordinate (D) at ($ (1,2.25) $);
	\coordinate (E) at ($ (2.5,2.25) $);
	\coordinate (F) at ($ (4,2.25) $);
	\coordinate (G) at ($ (1,3.5) $);
	\coordinate (H) at ($ (2.5,3.5) $);
	\coordinate (I) at ($ (4,3.5) $);

	\coordinate [label=below:$x_p$] (A1) at ($ (1,0) $);
	\coordinate [label=below:$x_{p+1}$] (B1) at ($ (4,0) $);
	\coordinate [label=left:$y_s$] (A2) at ($ (0,1) $);
	\coordinate [label=left:$y_{s+1}$] (C1) at ($ (0,3.5) $);

	\draw[-] (MX) -- (O);
	\draw[-] (MY) -- (O);
	\draw[->] (O) -- (X);
	\draw[->] (O) -- (Y);

	\draw[-, line width=1pt] (A) -- (C);
	\draw[-, line width=1pt] (A) -- (G);
	\draw[-, line width=1pt] (C) -- (I);
	\draw[-, line width=1pt] (G) -- (I);
	\draw[-, line width=1pt] (B) -- (H);
	\draw[-, line width=1pt] (D) -- (F);
	
	\fill [black] (A) circle (1.5pt);
	\fill [black] (B) circle (1.5pt);
	\fill [black] (C) circle (1.5pt);
	\fill [black] (D) circle (1.5pt);
	\fill [black] (E) circle (1.5pt);
	\fill [black] (F) circle (1.5pt);
	\fill [black] (G) circle (1.5pt);
	\fill [black] (H) circle (1.5pt);
	\fill [black] (I) circle (1.5pt);
	\draw[-,dashed,very thin] (A) -- (A1);
	\draw[-,dashed,very thin] (A) -- (A2);
	\draw[-,dashed,very thin] (C) -- (B1);
	\draw[-,dashed,very thin] (G) -- (C1);

	\node[right] at (5.5, 4) {$ h_x = x_{p+1}-x_p $};
	\node[right] at (5.5, 3) {$ h_y = y_{s+1}-x_s $};
	\node[right] at (5.5, 2) {$ x \in [x_p, x_{p+1}],\, y \in [y_s, y_{s+1}] $};
	\node[right] at (5.5, 1) {$ \Omega_{ps} = [x_p, x_{p+1}] \times [y_s, y_{s+1}] $};

	\draw[<->] ($(A)!0.5!(A1)$) -- ($(C)!0.5!(B1)$) node [midway, ann] {$h_x$};
	\draw[<->] ($(A)!0.5!(A2)$) -- ($(G)!0.5!(C1)$) node [midway, ann] {$h_y$};

	\draw[] ($(2.5, 2.25)$) -- ($(2.5, 2.25)$) node [midway, ann, scale=3] {$\Omega$};

	\draw[] (A) -- (A) node [midway, ann] {$1$};
	\draw[] (B) -- (B) node [midway, ann] {$2$};
	\draw[] (C) -- (C) node [midway, ann] {$3$};
	\draw[] (D) -- (D) node [midway, ann] {$4$};
	\draw[] (E) -- (E) node [midway, ann] {$5$};
	\draw[] (F) -- (F) node [midway, ann] {$6$};
	\draw[] (G) -- (G) node [midway, ann] {$7$};
	\draw[] (H) -- (H) node [midway, ann] {$8$};
	\draw[] (I) -- (I) node [midway, ann] {$9$};

\end{tikzpicture}
\end{center}

Значения функций в узлах:

\begin{center}
\begin{tabular}{ccc}
$\displaystyle \psi_1(x, y) = X_1(x) Y_1(y) $ &
$\displaystyle \psi_2(x, y) = X_2(x) Y_1(y) $ &
$\displaystyle \psi_3(x, y) = X_3(x) Y_1(y) $ \\

$\displaystyle \psi_4(x, y) = X_1(x) Y_2(y) $ &
$\displaystyle \psi_5(x, y) = X_2(x) Y_2(y) $ &
$\displaystyle \psi_6(x, y) = X_3(x) Y_2(y) $ \\

$\displaystyle \psi_7(x, y) = X_1(x) Y_3(y) $ &
$\displaystyle \psi_8(x, y) = X_2(x) Y_3(y) $ &
$\displaystyle \psi_9(x, y) = X_3(x) Y_3(y) $
\end{tabular}
\end{center}

И значение конечно-элементной аппроксимации на этом конечном элементе равно:
$$ u^{*}_{ps}(x, y) = \sum_{i=1}^9 q_i \psi_i(x, y)  $$

Аналитические выражения для вычисления элементов локальных матриц:

$$ G_{ij} = \myintegral{x_p}{x_{p+1}}\myintegral{y_s}{y_{s+1}} \lambda\roubr{\mypartial{\psi_i}{x}\mypartial{\psi_j}{x} + \mypartial{\psi_i}{y}\mypartial{\psi_j}{y}} \myd{x} \myd{y} $$

\begin{center}
$\displaystyle M_{ij}^\gamma = \myintegral{x_p}{x_{p+1}}\myintegral{y_s}{y_{s+1}} \gamma \psi_i \psi_j \myd{x} \myd{y} $, $\displaystyle \quad b_i = \myintegral{x_p}{x_{p+1}}\myintegral{y_s}{y_{s+1}} f \psi_i \myd{x} \myd{y} $
\end{center}

Вычисленные матрицы для билинейных прямоугольных элементов:
\setlength{\arraycolsep}{2.5pt} % default: 5pt
\medmuskip = 1mu
$$ \mb{G} = \frac{\lambda}{90}\frac{h_{x}}{h_{y}}\cdot\scalemath{0.5}{\left[\begin{matrix}28 & 14 & -7 & -32 & -16 & 8 & 4 & 2 & -1\\14 & 112 & 14 & -16 & -128 & -16 & 2 & 16 & 2\\-7 & 14 & 28 & 8 & -16 & -32 & -1 & 2 & 4\\-32 & -16 & 8 & 64 & 32 & -16 & -32 & -16 & 8\\-16 & -128 & -16 & 32 & 256 & 32 & -16 & -128 & -16\\8 & -16 & -32 & -16 & 32 & 64 & 8 & -16 & -32\\4 & 2 & -1 & -32 & -16 & 8 & 28 & 14 & -7\\2 & 16 & 2 & -16 & -128 & -16 & 14 & 112 & 14\\-1 & 2 & 4 & 8 & -16 & -32 & -7 & 14 & 28\end{matrix}\right]} + \frac{\lambda}{90}\frac{h_{y}}{h_{x}}\cdot\scalemath{0.5}{\left[\begin{matrix}28 & -32 & 4 & 14 & -16 & 2 & -7 & 8 & -1\\-32 & 64 & -32 & -16 & 32 & -16 & 8 & -16 & 8\\4 & -32 & 28 & 2 & -16 & 14 & -1 & 8 & -7\\14 & -16 & 2 & 112 & -128 & 16 & 14 & -16 & 2\\-16 & 32 & -16 & -128 & 256 & -128 & -16 & 32 & -16\\2 & -16 & 14 & 16 & -128 & 112 & 2 & -16 & 14\\-7 & 8 & -1 & 14 & -16 & 2 & 28 & -32 & 4\\8 & -16 & 8 & -16 & 32 & -16 & -32 & 64 & -32\\-1 & 8 & -7 & 2 & -16 & 14 & 4 & -32 & 28\end{matrix}\right]}
$$

\begin{center}
\begin{tabular}{cc}
\multirow{3}{*}{$\displaystyle \mb{C} = \gamma \frac{h_{x} h_{y}}{900}\cdot\scalemath{0.5}{\left[\begin{matrix}16 & 8 & -4 & 8 & 4 & -2 & -4 & -2 & 1\\8 & 64 & 8 & 4 & 32 & 4 & -2 & -16 & -2\\-4 & 8 & 16 & -2 & 4 & 8 & 1 & -2 & -4\\8 & 4 & -2 & 64 & 32 & -16 & 8 & 4 & -2\\4 & 32 & 4 & 32 & 256 & 32 & 4 & 32 & 4\\-2 & 4 & 8 & -16 & 32 & 64 & -2 & 4 & 8\\-4 & -2 & 1 & 8 & 4 & -2 & 16 & 8 & -4\\-2 & -16 & -2 & 4 & 32 & 4 & 8 & 64 & 8\\1 & -2 & -4 & -2 & 4 & 8 & -4 & 8 & 16\end{matrix}\right]} $} & $ \mb{M}^\gamma = \bar{\gamma}\mb{C} $ \\[1.25ex]
& $ \mb{f} = \roubr{f_1, f_2, f_3, f_4}^t $ \\[1.25ex]
& $ \mb{b} = \mb{C}\cdot \mb{f} $
\end{tabular}
\end{center}

Схема Кранка-Николсона:

\begin{center}
$\displaystyle \mypartial{u}{t} = \frac{u^j-u^{j-2}}{2\Delta t} $, $\displaystyle \quad \mypartialpow{u}{t}{2} = \frac{u^j-2u^{j-1}+u^{j-2}}{\Delta t^2} $
\end{center}

\begin{center}
$\displaystyle u = \frac{u^j+u^{j-2}}{2} $, $\displaystyle \quad f = \frac{f^j+f^{j-2}}{2} $
\end{center}

\begin{center}
$\displaystyle -\mydiv\roubr{\lambda \mygrad\frac{u^j+u^{j-2}}{2}} + \gamma \frac{u^j+u^{j-2}}{2} + \sigma \frac{u^j - u^{j-2}}{2\Delta t} + \chi \frac{u^j-2u^{j-1}+u^{j-2}}{\Delta t^2} = \frac{f^j+f^{j-2}}{2} $
\end{center}

Подставляя это в уравнение Галёркина, получаем СЛАУ из глобальных матриц:

$$ \roubr{\frac{\mb{G}}{2} + \frac{\mb{M}^\gamma}{2} + \frac{\mb{M}^\sigma}{2\Delta t} + \frac{\mb{M}^\chi}{\Delta t^2}}\mb{q}^j = \frac{\roubr{\mb{b}^j+\mb{b}^{j-2}}}{2} - \frac{\mb{G}\mb{q}^{j-2}}{2} - \frac{\mb{M}^\gamma\mb{q}^{j-2}}{2} + \frac{\mb{M}^\sigma \mb{q}^{j-2}}{2\Delta t} - \frac{\mb{M}^\chi \roubr{-2\mb{q}^{j-1} + \mb{q}^{j-2}}}{\Delta t^2} $$

В нашем случае, так как $\gamma$, $\sigma$, $\chi$ являются константами, можно записать:

$$ \roubr{\frac{\mb{G}}{2} + \mb{C}\roubr{\frac{\gamma}{2} + \frac{\sigma}{2\Delta t} + \frac{\chi}{\Delta t^2}}}\mb{q}^j = \frac{\roubr{\mb{b}^j+\mb{b}^{j-2}}}{2} - \frac{\mb{G}\mb{q}^{j-2}}{2} + \mb{C}\roubr{\mb{q}^{j-1}\frac{2\chi}{\Delta t^2} + \mb{q}^{j-2}\roubr{-\frac{\gamma}{2}+\frac{\sigma}{2\Delta t}-\frac{\chi}{\Delta t^2}}} $$

Для неравномерной же сетки по времени имеем только отличие в:

$$ t_2 = t^{j-2},\quad t_1 = t^{j-1},\quad t_0 = t^j $$

$$ \mypartial{u}{t} = \frac{u^j-u^{j-2}}{t_2-t_1} = \frac{u^j-u^{j-2}}{d_1} $$

$$ \mypartialpow{u}{t}{2} = 2\frac{u^j-u^{j-1}\cfrac{t_0-t_2}{t_1-t_2}+u^{j-2}\cfrac{t_0-t_1}{t_1-t_2}}{t_0\roubr{t_0-t_1-t_2} + t_1 t_2} = \frac{u^j-u^{j-1}m_1+u^{j-2}m_2}{d_2} $$

Эти выражения были упрощены при помощи замен:

$$ d_1 = t_0-t_2, \quad d_2 = \frac{t_0\roubr{t_0-t_1-t_2} + t_1 t_2}{2}, \quad m_1 = \frac{t_0-t_2}{t_1-t_2}, \quad m_2 = \frac{t_0-t_1}{t_1-t_2} $$ 

И итоговый результат будет:

$$ \roubr{\frac{\mb{G}}{2} + \mb{C}\roubr{\frac{\gamma}{2} + \frac{\sigma}{d_1} + \frac{\chi}{d_2}}}\mb{q}^j = \frac{\roubr{\mb{b}^j+\mb{b}^{j-2}}}{2} - \frac{\mb{G}\mb{q}^{j-2}}{2} + \mb{C}\roubr{\mb{q}^{j-1}\frac{m_1\chi}{d_2} + \mb{q}^{j-2}\roubr{-\frac{\gamma}{2}+\frac{\sigma}{d_1}-\frac{m_2\chi}{d_2}}} $$

\section{Структуры данных}

\noindent\textbf{Для задания сетки используется класс:}

\begin{tcolorbox}[
	colback=white,
	colframe=black!30!white,
	boxrule=0.5pt, 
	listing only,
	left=-0.5mm,
	leftrule=4mm,
	arc=2mm, outer arc=2mm,
	top=0pt,
	bottom=0pt,
	enhanced jigsaw,
	coltitle=black, 
	fonttitle=\bfseries\ttfamily,
	break at=-\baselineskip/0pt/\textheight, 
]
\begin{minted}[
	breaklines,
	breakanywhere=true,
	autogobble,
	linenos,
	numbersep=1mm,
	mathescape, 
	fontsize=\fontsize{8}{8}\selectfont, 
	tabsize=4
]{c++}
class grid_generator_t
{
public:
	grid_generator_t(double a, double b, int n, double t = 0);
	double operator()(int i) const;
	int size(void) const;
	double back(void) const;
private:
	double a, len, t, n1;
};
\end{minted}
\end{tcolorbox} 

\noindent\textbf{Для задания узла конечного элемента структура:}

\begin{tcolorbox}[
	colback=white,
	colframe=black!30!white,
	boxrule=0.5pt, 
	listing only,
	left=-0.5mm,
	leftrule=4mm,
	arc=2mm, outer arc=2mm,
	top=0pt,
	bottom=0pt,
	enhanced jigsaw,
	coltitle=black, 
	fonttitle=\bfseries\ttfamily,
	break at=-\baselineskip/0pt/\textheight, 
]
\begin{minted}[
	breaklines,
	breakanywhere=true,
	autogobble,
	linenos,
	numbersep=1mm,
	mathescape, 
	fontsize=\fontsize{8}{8}\selectfont, 
	tabsize=4
]{c++}
struct basic_elem_t
{
	int i; /// Номер узла

	double x, y; /// Координаты узла

	basic_elem_t *up, *down, *left, *right; /// Указатели на соседей узла

	/** Проверяет, является ли элемент граничным. Он таким явлется, если у него нет хотя бы одного соседа. */
	bool is_boundary(void) const;
};
\end{minted}
\end{tcolorbox} 

\noindent\textbf{Для задания конечного элемента используется структура:}

\begin{tcolorbox}[
	colback=white,
	colframe=black!30!white,
	boxrule=0.5pt, 
	listing only,
	left=-0.5mm,
	leftrule=4mm,
	arc=2mm, outer arc=2mm,
	top=0pt,
	bottom=0pt,
	enhanced jigsaw,
	coltitle=black, 
	fonttitle=\bfseries\ttfamily,
	break at=-\baselineskip/0pt/\textheight, 
]
\begin{minted}[
	breaklines,
	breakanywhere=true,
	autogobble,
	linenos,
	numbersep=1mm,
	mathescape, 
	fontsize=\fontsize{8}{8}\selectfont, 
	tabsize=4
]{c++}
struct elem_t
{
    int i; /// Номер конечного элемента
    basic_elem_t* e[9]; /// Указатели на все 9 элементов конечного узла, нумерация такая:
    /** 
         Y
         ^
         |  7-----8-----9
         |  |     |     |
         |  4-----5-----6
         |  |     |     |
         |  1-----2-----3
         |
        -+---------------> X
         |
     */

    double get_hx(void) const; /// Ширина конечного элемента
    double get_hy(void) const; /// Высота конечного элемента

    /** Рассчитать значение внутри конечного элемента. q - вектор рассчитыванных весов. */
    double value(double x, double y, const vector_t& q) const;
};
\end{minted}
\end{tcolorbox} 

\noindent\textbf{Прямоугольная сетка задается и вычисляется с помощью класса:}

\begin{tcolorbox}[
	colback=white,
	colframe=black!30!white,
	boxrule=0.5pt, 
	listing only,
	left=-0.5mm,
	leftrule=4mm,
	arc=2mm, outer arc=2mm,
	top=0pt,
	bottom=0pt,
	enhanced jigsaw,
	coltitle=black, 
	fonttitle=\bfseries\ttfamily,
	break at=-\baselineskip/0pt/\textheight, 
]
\begin{minted}[
	breaklines,
	breakanywhere=true,
	autogobble,
	linenos,
	numbersep=1mm,
	mathescape, 
	fontsize=\fontsize{8}{8}\selectfont, 
	tabsize=4
]{c++}
class grid_t
{
public:
    vector<elem_t> es; /// Массив конечных элементов сетки
    vector<basic_elem_t> bes; /// Массив узлов сетки
    int n; /// Число узлов

    /** Рассчитать неравномерную сетку. */
    void calc(const grid_generator_t& gx, const grid_generator_t& gy);
};
\end{minted}
\end{tcolorbox} 

\noindent\textbf{Локальные матрицы формируются, получая на вход конечный элемент \texttt{elem\_t}.}

\noindent\textbf{Для генерации разреженной матрицы используется класс с возможностью произвольного доступа к элементам:}

\begin{tcolorbox}[
	colback=white,
	colframe=black!30!white,
	boxrule=0.5pt, 
	listing only,
	left=-0.5mm,
	leftrule=4mm,
	arc=2mm, outer arc=2mm,
	top=0pt,
	bottom=0pt,
	enhanced jigsaw,
	coltitle=black, 
	fonttitle=\bfseries\ttfamily,
	break at=-\baselineskip/0pt/\textheight, 
]
\begin{minted}[
	breaklines,
	breakanywhere=true,
	autogobble,
	linenos,
	numbersep=1mm,
	mathescape, 
	fontsize=\fontsize{8}{8}\selectfont, 
	tabsize=4
]{c++}
class matrix_sparse_ra_t
{
public:
    matrix_sparse_ra_t(int n);

    /** Установить значение в позиции (i, j) */
    double& operator()(int i, int j);

    /** Получить значение в позиции (i, j). Если туда ещё не устанавливалось значение, вызывается исключение. */
    const double& operator()(int i, int j) const;

    /** Преобразует текущую матрицу к разреженной матрице. */
    matrix_sparse_t to_sparse(void) const;
private:
    int n;
    vector<double> dm;
    vector<map<int, double>> lm, um;
};
\end{minted}
\end{tcolorbox} 


\section{Исследования}

Во всех исследованиях заданы следующие параметры $ \lambda = \gamma = \sigma = \chi = 1$.

СЛАУ решается при помощи Локально-Оптимальной Схемы (ЛОС) с неполным LU предобуславливанием.

\subsection{Таблицы}

Далее в таблицах будут указаны две функции: $\operatorname{space}(x, y)$ и $\operatorname{time}(t)$, итоговая функция $u$ будет формироваться из них: $u(x, y, t) = \operatorname{space}(x, y) + \operatorname{time}(t) $.

В таблицах для каждой функции указано три значения: 

\begin{easylist}
\ListProperties(Hang1=true, Margin2=12pt, Style1**=$\bullet$ , Hide1=1)
& Интеграл разности между истинной функцией и конечно-элементоной аппроксимацией.
& Норма разности векторов $q$ для найденного решения и $q$, полученного из истинного значения функции.
& Время решения в миллисекундах.
\end{easylist}



\newcommand{\mytable}[1]{
\begin{center}
\noindent\pgfplotstabletypeset[
	columns={a,$0$,$t$,$t^2$,$t^3$,$t^4$,$e^t$},
	columns/a/.style={string type, column name={\backslashbox{$\operatorname{space}(x, y)$}{$\operatorname{time}(t)$}}},
	columns/$0$/.style={string type},
	columns/$t$/.style={string type},
	columns/$t^2$/.style={string type},
	columns/$t^3$/.style={string type},
	columns/$t^4$/.style={string type},
	columns/$e^t$/.style={string type, column type/.add={}{|},},
	every head row/.style={before row=\hline,after row=\hline\hline}, 
	every last row/.style={after row=\hline},
	column type/.add={|}{},
	col sep=tab,
]{#1.txt}
\end{center}
}

\subsubsection{10 на 10 на 10}

Сетка по пространству: $ (x, y) \in [0, 1] \times [0, 1] $, строк и столбцов $10$. Сетка по времени: $ t \in [0, 1] $, количество элементов сетки 10. Все сетки равномерные.

\mytable{functions_table_10_10_10}

\conclusion полностью (на всей области конечных элементов, а не только в узлах) аппроксимируются линейные и \textbf{квадратичные} функции по пространству и для степени $t$ равной 0, 1 или 2.

\conclusion значения в узлах полностью аппроксимируются по пространству только до полиномов 4 степени включительно.

\conclusion порядок аппроксимации по пространству --- $4$, порядок аппроксимации по времени --- $2$.

\conclusion все функции считаются примерно за одинаковое время.

\subsubsection{50 на 50 на 50}

Сетки аналогичны предыдущему пункту, только число элементов по всем сеткам равно 50.

\mytable{functions_table_50_50_50}

\conclusion предыдущие выводы не опровеглись.

\conclusion время вычислений выросло примерно в 125 раз.

\subsection{Неравномерные сетки}

\subsubsection{Функции нелинейной сетки}

В ходе выполнения лабораторной работы была обнаружена функция, позволяющая легко задавать неравномерную сетку, сгущающуюся к одному из концов.

Если у нас задано начало --- $a$ и конец сетки --- $b$, а количество элементов $n$, тогда сетку можно задать следующим образом:

$$ x_i = a + m\left(\frac{i}{n}\right) \cdot (b-a), i=\overline{0, n} $$

где $m(x)$ --- некоторая функция, задающая неравномерную сетку. При этом $x$ обязан принадлежать области $[0, 1]$, а функция $m$ возвращать значения из той же области, и при этом быть монотонной на этом участке. Тогда гарантируется условие монотонности сетки, то есть что при $j \leqslant i\, \Rightarrow\, x_j \leqslant x_i$. 

\textit{Пример:} при $m(x) = x$, сетка становится равномерной.

Найденная функция зависят от параметра неравномерности t:

$$ m_t(x) = \frac{1-\roubr{1-|t|}^{x\operatorname{sign} t}}{1-\roubr{1-|t|}^{\operatorname{sign} t}} $$

Эта функции вырождается в $x$ при $t=0$; при $t=-1$, она вырождается в сетку, полностью находящуюся в $0$; а при $t=1$ она полностью сгущается к $1$. 

Таким образом, можно исследовать различные неравномерные сетки, изменяя параметр от $-1$ до $1$, где точка $t=0$ будет являться результатом на равномерной сетке.

\subsubsection{По пространству}

Сетка по пространству: $ (x, y) \in [0, 1] \times [0, 1] $, строк и столбцов $10$. Сетка по времени: $ t \in [0, 1] $, количество элементов сетки 10.

\newcommand{\includetwoimages}[1]{
\noindent\includegraphics[width=.49\textwidth]{#1_norm.png}\includegraphics[width=.49\textwidth]{#1_integral.png}
}

\myparagraph{Функция 1}
$$ u = x^2 + y^2 + t^2 $$

\includetwoimages{space_tgrid_0}

\conclusion так как эта функция полностью аппроксимируется данным методом в узлах, то не имеет значения насколько сетка неравномерна, примерно во всех элементах она имеет одинаковую невязку, согласно левому графику. Разве что в сильно неравномерных сетках, где элементы сильно сгущены к одному из концов, точностью страдает на несколько порядков.

\conclusion а по интегральной норме лучшей сеткой является раномерная сетка согласно правому графику.

\myparagraph{Функция 2}
$$ u = x^4 + y^3x + t^4 $$

\includetwoimages{space_tgrid_1}

\conclusion согласно левому графику норма в узлах лучше всего аппроксимируется при сгущении по $y$ в одну или другую сторону. По $x$ же неравномерность сетки практически ни на что не влияет.

\conclusion лучшая точность, даваемая неравномерной сетки примерно на полпорядка лучше, чем при равномерной.

\conclusion по интегральной же норме существует некоторая комбинация параметров, при которых сетка получается оптимальной. Но различия от неравномерной сетки ничтожны.

\myparagraph{Функция 3}
$$ u = e^{xy} + e^{t^2} $$

\includetwoimages{space_tgrid_2}

\conclusion согласно левому графику аппроксимация в узлах тоже имеет некоторые оптимальлные значения, причем точность увеличивается на порядок.

\conclusion для интегральной же нормы различия же от равномерной сетки ничтожны при любых параметрах сетки.

\myparagraph{Функция 4}
$$ u = e^{(1-x)(1-y)} + e^{(1-t)^2} $$

\includetwoimages{space_tgrid_3}

\conclusion эта функция отличается от предыдущей, что для неё инвертировано положение $x$ и $y$, график по интегральной норме соответственно изменился.

\myparagraph{Функция 5}
$$ u = x^3 + y^4 x^2 t + t^2 e^t $$

\includetwoimages{space_tgrid_4}

\conclusion всё аналогично предыдущим выводам и функциям.

\myparagraph{Общие выводы}

\conclusion хорошая аппроксимация в узлах $\neq$ хорошая аппроксимация по интегральной норме.

\conclusion согласно интегральной норме для неполиномиальных функций существует некоторый набор параметров $t_x$ и $t_y$, при которых нелинейная сетка оптимальным образом аппроксимирует функцию.

\conclusion согласно норме в узлах для неполиномиальных функций оптимальными являются параметры в окрестности $\pm 1$.

\subsubsection{По времени}

Сетка по пространству: $ (x, y) \in [0, 1] \times [0, 1] $, строк и столбцов $10$. Сетка по времени: $ t \in [0, 1] $, количество элементов сетки 10.

\newcommand{\includetwographs}[1]{
\noindent\begin{tikzpicture}
\begin{semilogyaxis}[xlabel=t,ylabel=Интегральная разность,width=0.49\textwidth, height=6cm]
\addplot[black, dashed, no markers] table [y=integral, x=t]{#1.txt};
\addplot[black, no markers] table [y=uniform_integral, x=t]{#1.txt};
\legend{Неравномерная сетка,Равномерная сетка}
\end{semilogyaxis}
\end{tikzpicture}
\begin{tikzpicture}
\begin{semilogyaxis}[xlabel=t,ylabel=Норма разности векторов,width=0.49\textwidth, height=6cm]
\addplot[black, dashed, no markers] table [y=norm, x=t]{#1.txt};
\addplot[black, no markers] table [y=uniform_norm, x=t]{#1.txt};
\legend{Неравномерная сетка,Равномерная сетка}
\end{semilogyaxis}
\end{tikzpicture}
}

\myparagraph{Функция 1}
$$ u = x^2 + y^2 + t^2 $$

\includetwographs{time_tgrid_0}

\conclusion так как по времени эта функция аппроксимируется точно, то неравномерность сетки никак не влияет на точность. Правый график колеблется в пределах максимальной точности, левый же абсолютно не меняется.

\myparagraph{Функция 2}
$$ u = x^4 + y^3x + t^4 $$

\includetwographs{time_tgrid_1}

\conclusion для данной функции в сетки есть выраженный минимум в окрестности $t=0.7$, но улучшение точности на нем примерно полпорядка.

\myparagraph{Функция 3}
$$ u = e^{xy} + e^{t^2} $$

\includetwographs{time_tgrid_2}

\conclusion всё аналогично предыдущему.

\myparagraph{Функция 4}
$$ u = e^{(1-x)(1-y)} + e^{(1-t)^2} $$

\includetwographs{time_tgrid_3}

\conclusion эта функция является перевернутой версией предыдущей, но график аналогчно не переверлся, а наблюдается более сложная зависимость. Для данной функции неравномерная сетка по времени практичеки везде дает отрицательный эффект по сравнению с равномерной сеткой.

\myparagraph{Функция 5}
$$ u = x^3 + y^4 x^2 t + t^2 e^t $$

\includetwographs{time_tgrid_4}

\myparagraph{Общие выводы}

\conclusion у множества функций наблюдалось схожее поведение на неравномерной сетке по времени, с наличием ярко выраженного минимума, и использование сетки с данным оптимальным параметром может улучшить точность решения на полпорядка.

\subsection{Порядок сходимости}

Исследуется на функции:
$$ u(x, y, t) = e^{xy} + e^{t^2} $$

\subsubsection{Увеличение размерности по пространству}

Сетка по пространству: $ (x, y) \in [0, 1] \times [0, 1] $. Сетка по времени: $ t \in [0, 1] $, количество элементов сетки 200.

\noindent\begin{tikzpicture}
\begin{semilogyaxis}[xlabel=Число узлов,ylabel=Интегральная разность,width=\textwidth, height=6cm]
\addplot[black, no markers] table [y=integral, x=i]{space_sgrid.txt};
\addplot[black, dashed, smooth, domain=49:5776] {529^(0.75)*x^(-0.75)*0.000190175};
\legend{Точность решения,$c\cdot x^{-0.75}$,}
\end{semilogyaxis}
\end{tikzpicture}

\conclusion согласно интегральной норме, порядок сходимости $\approx 0.75$.

\noindent\begin{tikzpicture}
\begin{semilogyaxis}[xlabel=Число узлов,ylabel=Норма разности векторов,width=\textwidth, height=6cm]
\addplot[black, no markers] table [y=norm, x=i]{space_sgrid.txt};
\addplot[black, dashed, smooth, domain=49:5776] {529^(0.45)*x^(-0.45)*8.59283*10^(-7)};
\legend{Точность решения,$c\cdot x^{-0.45}$,}
\end{semilogyaxis}
\end{tikzpicture}

\conclusion согласно норме разности векторов, порядок сходимости $\approx 0.45$.

\noindent\begin{tikzpicture}
\begin{axis}[xlabel=Число узлов,ylabel=Время решения мкс,width=\textwidth, height=6cm]
\addplot[black, no markers] table [y=time, x=i]{space_sgrid.txt};
\addplot[black, dashed, smooth, domain=49:5776] {x/5776*4.89955*10^7};
\end{axis}
\end{tikzpicture}

\conclusion время решения почти линейно зависит от числа узлов (с учетом погрешности, вносимой многопоточностью).

\subsubsection{Увеличение размерности по времени}

Сетка по пространству: $ (x, y) \in [0, 1] \times [0, 1] $, строк и столбцов $20$. Сетка по времени: $ t \in [0, 1] $.

\noindent\begin{tikzpicture}
\begin{semilogyaxis}[xlabel=Число временных итераций,ylabel=Интегральная разность,width=\textwidth, height=6cm]
\addplot[black, no markers] table [y=integral, x=i]{time_sgrid.txt};
%\addplot[black, dashed, smooth, domain=49:5776] {529^(0.75)*x^(-0.75)*0.000190175};
\legend{Точность решения,$c\cdot x^{-0.75}$,}
\end{semilogyaxis}
\end{tikzpicture}

\conclusion при увеличении числа итераций по времени, увеличивается интегральная норма, но потом она достигает предельного значения и далее не увеличивается, потому что количество элементов сетки неизменно, и они не способны точно аппроксимировать данную функцию, поскольку она непредставима в линейных элементах.

\noindent\begin{tikzpicture}
\begin{semilogyaxis}[xlabel=Число временных итераций,ylabel=Норма разности векторов,width=\textwidth, height=6cm]
\addplot[black, no markers] table [y=norm, x=i]{time_sgrid.txt};
\addplot[black, dashed, smooth, domain=3:501] {100^(2)*x^(-2)*9.10204*10^(-6)};
\legend{Точность решения,$c\cdot x^{-2}$,}
\end{semilogyaxis}
\end{tikzpicture}

\conclusion при увеличении числа итераций по времени, увеличивается точность, и порядок аппроксимации по времени равен 2.

\noindent\begin{tikzpicture}
\begin{axis}[xlabel=Число временных итераций,ylabel=Время решения мкс,width=\textwidth, height=6cm]
\addplot[black, no markers] table [y=time, x=i]{time_sgrid.txt};
\addplot[black, dashed, smooth, domain=3:501] {x/501*6*10^6};
\end{axis}
\end{tikzpicture}

\conclusion время решения почти линейно зависит от числа итераций по времени (с учетом погрешности, вносимой многопоточностью).

\section{Код}

\subsection{Аналитический расчёт формул}

Для расчётов формул была использована библиотека \texttt{sympy} на \texttt{Python}, благодаря ей были вычислены матрицы $G$, $C$ размера $9\times 9$; и билинейные базисные функции. 

В следующем файле с вычисляются эти функции и матрицы для билинейных базисов:

\mycodeinput{python}{../integrate_fem_2.py}{integrate\_fem\_2.py}

А здесь вычисляются для биквадратичных базисов:

\mycodeinput{python}{../integrate_fem_3.py}{integrate\_fem\_3.py}

Программу вполне можно обобщить для любых би-$n$-базисов.

\subsection{Файлы заголовков}

\mycodeinput{c++}{../lib.h}{lib.h}
\mycodeinput{c++}{../sparse.h}{sparse.h}
\mycodeinput{c++}{../fem.h}{fem.h}

\subsection{Исходные файлы}

\mycodeinput{c++}{../sparse.cpp}{sparse.cpp}
\mycodeinput{c++}{../fem.cpp}{fem.cpp}

\subsection{Исследования}

\mycodeinput{c++}{../main.cpp}{main.cpp}

\subsection{Визуализация}

\mycodeinput{python}{../plot.py}{plot.py}