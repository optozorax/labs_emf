\newcommand{\ltcell}[1]{\begin{tabular}{@{}l@{}}#1\end{tabular}}

% Собственные обозначения для дополнительной нумерации глав
\setcounter{secnumdepth}{4} % Чтобы параграф отображался цифрами
\newcommand{\myparagraph}[1]{\paragraph{#1}\mbox{}\\} % Аналог \subsubsubsection

% Собственные обозначения для скобок
\newcommand{\roubr}[1]{\left(#1\right)}  % round  brackets ()
\newcommand{\sqbr}[1]{\left[#1\right]}   % sqaure brackets [] 
\newcommand{\cubr}[1]{\left\{#1\right\}} % curtly brackets {}

% Жирный текст в математике
\newcommand{\mb}[1]{\mathbf{#1}}

\newcommand{\myd}[1]{\operatorname{d}\!#1}

% Дивергенция и градиент
\newcommand{\mydiv}[0]{\operatorname{div}}
\newcommand{\mygrad}[0]{\operatorname{grad}}
\newcommand{\mypartial}[2]{\frac{\partial #1}{\partial #2}}
\newcommand{\mypartialpow}[3]{\frac{\partial^{#3} #1}{\partial #2^{#3}}}
\newcommand{\myintegral}[2]{\int\limits_{#1}^{#2}}

Решаемое уравнение в общем виде в декартовой системе координат:

$$ -\mydiv\roubr{\lambda\mygrad u} + \gamma u + \sigma \mypartial{u}{t} + \chi \mypartialpow{u}{t}{2} = f $$

Первые краевые условия:

Эквивалентная постановка в форме уравнения Галёркина:

Аппроксимация уравнения Галёркина на конечномерных подпространствах:

Формулы для билинейных базисных функций прямоугольных элементов:

Аналитические выражения для вычисления элементов локальных матриц:

$$ G_{ij} = \myintegral{x_p}{x_{p+1}}\myintegral{y_s}{y_{s+1}} \lambda\roubr{\mypartial{\psi_i}{x}\mypartial{\psi_j}{x} + \mypartial{\psi_i}{y}\mypartial{\psi_j}{y}} \myd{x} \myd{y} $$

$$ M_{ij}^\gamma = \myintegral{x_p}{x_{p+1}}\myintegral{y_s}{y_{s+1}} \gamma \psi_i \psi_j \myd{x} \myd{y} $$

$$ b_i = \myintegral{x_p}{x_{p+1}}\myintegral{y_s}{y_{s+1}} f \psi_i \myd{x} \myd{y} $$

$$ \mb{G} = \frac{\bar{\lambda}}{6}\frac{h_y}{h_x}\begin{pmatrix}
2 & -2 & 1 & -1 \\
-2 & 2 & -1 & 1 \\
1 & -1 & 2 & -2 \\
-1 & 1 & -2 & 2
\end{pmatrix} + \frac{\bar{\lambda}}{6}\frac{h_x}{h_y}\begin{pmatrix}
2 & 1 & -2 & -1 \\
1 & 2 & -1 & -2 \\
-2 & -1 & 2 & 1 \\
-1 & 2 & 1 & 2
\end{pmatrix} $$

$$ \mb{C} = \frac{h_x h_y}{36}\begin{pmatrix}
4 & 2 & 2 & 1 \\
2 & 4 & 1 & 2 \\
2 & 1 & 4 & 2 \\
1 & 2 & 2 & 4
\end{pmatrix} $$

$$ \mb{M}^\gamma = \bar{\gamma}\mb{C} $$

$$ \mb{f} = \roubr{f_1, f_2, f_3, f_4}^t $$

$$ \mb{b} = \mb{C}\cdot \mb{f} $$




Схема Кранка-Николсона:

$$ \mypartial{u}{t} = \frac{u^j-u^{j-2}}{2\Delta t} $$

$$ \mypartialpow{u}{t}{2} = \frac{u^j-2u^{j-1}+u^{j-2}}{\Delta t^2} $$

$$ u = \frac{u^j+u^{j-2}}{2} $$

$$ f = \frac{f^j+f^{j-2}}{2} $$

$$ -\mydiv\roubr{\lambda \mygrad\frac{u^j+u^{j-2}}{2}} + \gamma \frac{u^j+u^{j-2}}{2} + \sigma \frac{u^j - u^{j-2}}{2\Delta t} + \chi \frac{u^j-2u^{j-1}+u^{j-2}}{\Delta t^2} = \frac{f^j+f^{j-2}}{2} $$

Подставляя это в уравнение Галёркина, получаем СЛАУ:

$$ \roubr{\frac{\mb{G}}{2} + \frac{\mb{M}^\gamma}{2} + \frac{\mb{M}^\sigma}{2\Delta t} + \frac{\mb{M}^\chi}{\Delta t^2}}\mb{q}^j = \frac{\roubr{\mb{b}^j+\mb{b}^{j-2}}}{2} - \frac{\mb{G}\mb{q}^{j-2}}{2} - \frac{\mb{M}^\gamma\mb{q}^{j-2}}{2} + \frac{\mb{M}^\sigma \mb{q}^{j-2}}{2\Delta t} - \frac{\mb{M}^\chi \roubr{-2\mb{q}^{j-1} + \mb{q}^{j-2}}}{\Delta t^2} $$

В нашем случае, так как $\gamma$, $\sigma$, $\chi$ являются константами, можно записать:

$$ \roubr{\frac{\mb{G}}{2} + \mb{C}\roubr{\frac{\gamma}{2} + \frac{\sigma}{2\Delta t} + \frac{\chi}{\Delta t^2}}}\mb{q}^j = \frac{\roubr{\mb{b}^j+\mb{b}^{j-2}}}{2} - \frac{\mb{G}\mb{q}^{j-2}}{2} + \mb{C}\roubr{\mb{q}^{j-1}\frac{2\chi}{\Delta t^2} + \mb{q}^{j-2}\roubr{-\frac{\gamma}{2}+\frac{\sigma}{2\Delta t}-\frac{\chi}{\Delta t^2}}} $$

Для неравномерной же сетки по времени имеем только отличие в:

$$ t_2 = t^{j-2},\quad t_1 = t^{j-1},\quad t_0 = t^j $$

$$ \mypartial{u}{t} = \frac{u^j-u^{j-2}}{t_2-t_1} = \frac{u^j-u^{j-2}}{d_1} $$

$$ \mypartialpow{u}{t}{2} = 2\frac{u^j-u^{j-1}\frac{t_0-t_2}{t_1-t_2}+u^{j-2}\frac{t_0-t_1}{t_1-t_2}}{t_0\roubr{t_0-t_1-t_2} + t_1 t_2} = \frac{u^j-u^{j-1}m_1+u^{j-2}m_2}{d_2} $$

Эти выражения были упрощены при помощи замен:

$$ d_1 = t_2-t_1, \quad d_2 = \frac{t_0\roubr{t_0-t_1-t_2} + t_1 t_2}{2}, \quad m_1 = \frac{t_0-t_2}{t_1-t_2}, \quad m_2 = \frac{t_0-t_1}{t_1-t_2} $$ 

И итоговый результат будет:

$$ \roubr{\frac{\mb{G}}{2} + \mb{C}\roubr{\frac{\gamma}{2} + \frac{\sigma}{d_1} + \frac{\chi}{d_2}}}\mb{q}^j = \frac{\roubr{\mb{b}^j+\mb{b}^{j-2}}}{2} - \frac{\mb{G}\mb{q}^{j-2}}{2} + \mb{C}\roubr{\mb{q}^{j-1}\frac{m_1\chi}{d_2} + \mb{q}^{j-2}\roubr{-\frac{\gamma}{2}+\frac{\sigma}{d_1}-\frac{m_2\chi}{d_2}}} $$

% a = 0.5 * m + (cs.chi/dt/dt + cs.sigma/2.0/dt) * c + 0.5 * g;
% b = 0.5 * (b0 + bll) - m * (cs.chi/dt/dt * (-2.0*ql + qll) - cs.sigma/2.0/dt * qll) - 0.5*g * qll - 0.5*m * qll;