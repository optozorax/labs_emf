\pgfplotstableset{
	begin table={\rowcolors{2}{gray!25}{white}\begin{tabular}},
	end table=\end{tabular},
}

\mytitlepage{прикладной математики}{3, 4}{Уравнения математической физики}{\tcell{Решение гармонических задач\\Решение несимметричных СЛАУ}}{ПМ-63}{Шепрут И.И.}{1, 9}{Патрушев И.И.}{2019}

\section{Цель работы}

Разработать программу решения гармонической задачи методом конечных элементов. Провести сравнение прямого и итерационного методов решения получаемой в результате конечноэлементной аппроксимации СЛАУ.

Изучить особенности реализации итерационных методов BCG, BCGStab, GMRES для СЛАУ с несимметричными разреженными матрицами. Исследовать влияние предобусловливания на сходимость этих методов.

\section{Задание}

\begin{easylist}
\ListProperties(Hang=true, Margin2=12pt, Style2**=$\bullet$ , Hide2=2, Hide1=0)
& Выполнить конечноэлементную аппроксимацию исходного уравнения в соответствии с заданием. Получить формулы для вычисления компонент матрицы  A и вектора правой части  b .
& Реализовать программу решения гармонической задачи с учетом следующих требований:
&& язык программирования С++ или Фортран;
&& предусмотреть возможность задания неравномерной сетки по пространству, разрывность параметров уравнения по подобластям, учет краевых условий; 
&& матрицу хранить в разреженном строчном формате с возможностью перегенерации ее в профильный формат;
&& реализовать (или воспользоваться реализованными в курсе «Численные метды») методы решения СЛАУ: итерационный --- локально-оптимальную схему или метод сопряженных градиентов для несимметричных матриц с предобуславливанием и прямой --- LU -разложение или его модификации [2, стр. 871], [3].
& Протестировать разработанную программу на полиномах первой степени.
& Провести исследования реализованных методов для сеток с небольшим количеством узлов 500-1000 и большим количеством узлов --- порядка 20000-50000 для различных значений параметров: $10^{-4}\leqslant \omega\leqslant 10^9$, $10^{2}\leqslant \lambda\leqslant 8\cdot 10^5$, $0\leqslant \sigma\leqslant 10^8$, $8.81\cdot 10^{-12}\leqslant \chi\leqslant 10^{-10}$. Для всех решенных задач сравнить вычислительные затраты, требуемые для решения СЛАУ итерационным и прямым методом.
\end{easylist}

\noindent\textbf{Лабораторная работа \textnumero 3. Вариант 1:} Решить одномерную гармоническую задачу в декартовых координатах, базисные функции – линейные.

\noindent\textbf{Лабораторная работа \textnumero 4. Вариант 9:} Реализовать решение СЛАУ методом BSGSTAB с LU-предобуславливанием.

\section{Исследования}

Далее под нормой решения будет подразумеваться следующее значение: $\int_{-\infty}^{\infty} (u^s(x)-u^{s*}(x)) dx + \int_{-\infty}^{\infty} (u^c(x)-u^{c*}(x)) dx$, где $u^s$, $u^c$ --- истинные функции, а $u^{s*}$, $u^{c*}$ --- их конечноэлементные аппроксимации.

\subsection{Изменение констант}

Исследования проводились для функций: $u^s(x) = 3x$, $u^c(x) = -10x$, на отрезке: $x \in [1, 2]$, с начальными значениями констант: $\omega=1$, $\lambda=1$, $\sigma=1$, $\chi=10^{-11}$. Требуемая невязка при решении СЛАУ: $\varepsilon=10^{-16}$. 

Методы для решения СЛАУ: ЛОС с LU предобуславливанием; BSGSTAB с LU предобуславливанием. 

Время указано в микросекундах.

\newcommand{\bigtable}[1]{
\begin{center}
\noindent\pgfplotstabletypeset[
	columns/param/.style={string type, column name={Параметр}},
	columns/los_norm/.style={column name={LOS норма}},
	columns/bsg_norm/.style={column name={BSG норма}},
	columns/los_time/.style={column name={\tcell{LOS\\время}}},
	columns/bsg_time/.style={column name={\tcell{BSG\\время}}},
	columns/los_iter/.style={column name={\tcell{LOS\\итераций}}},
	columns/bsg_iter/.style={column name={\tcell{BSG\\итераций}}, column type/.add={}{|},},
	every head row/.style={before row=\hline,after row=\hline\hline}, 
	every last row/.style={after row=\hline},
	column type/.add={|}{},
	col sep=tab,
]{#1.txt}
\end{center}
}

\newcommand{\bigtables}[1]{
\bigtable{#1_omega}
\bigtable{#1_lambda}
\bigtable{#1_sigma}
\bigtable{#1_xi}
}

\subsubsection{Размер сетки 100}

\bigtables{3_parameters_100}

\noindent\textbf{Вывод:} LOS работает быстрее, чем BSG.

\noindent\textbf{Вывод:} в среднем у BSG меньше итераций, чем у LOS.

\noindent\textbf{Вывод:} норма решения лучше у ЛОС, чем у BSG.

\noindent\textbf{Вывод:} при увеличении $\omega$ норма улучшается, однако до некоторого предела $\omega = 10^8$, после которого метод расходится.

\noindent\textbf{Вывод:} при увеличении $\lambda$ норма для ЛОС никак не меняется, когда для BSG она ухудшается.

\noindent\textbf{Вывод:} при увеличении $\sigma$ норма повышается, и достигает экстремума в $\sigma = 10^6$.

\noindent\textbf{Вывод:} изменение $\chi$ в допустимых пределах ни на что особо не влияет.

\subsubsection{Размер сетки 50000}

\bigtables{3_parameters_50000}

\noindent\textbf{Вывод:} при увеличении $\omega$ норма улучшается.

\noindent\textbf{Вывод:} норма при изменении $\lambda$ колеблется сложным образом без явных экстремумов или монотонностей.

\noindent\textbf{Вывод:} при увеличении $\sigma$ норма повышается монотонно.

\noindent\textbf{Вывод:} время решения увеличилось примерно в 1000 раз, когда количество элементов только в 500.

\noindent Остальные выводы подтвердились.

\subsection{Зависимость точности от нелинейной сетки}

Неравномерная функция рассчитывается аналогично предыдущей работе, используются функции $m_{3, t}(x)$, $m_{4, t}(x)$. Красным обозначается сетка, сгущающаяся к началу, синим к концу.

Исследования проводились на отрезке: $x \in [1, 2]$, с значениями констант: $\omega=1$, $\lambda=1$, $\sigma=1$, $\chi=1$. Требуемая невязка при решении СЛАУ: $\varepsilon=10^{-16}$. Число элементов сетки: $50$.

\textit{Замечание:} так как точность решения или <<норма>> рассчитывается при помощи интеграла, то неравномерная сетка может более точно аппроксимировать значения функции в точках узлов, но менее точно делать это между узлами, и так как при неравномерной сетке пространство между узлами сильно увеличивается для некоторых значений $t$.

\newcommand{\graphgrid}[1]{
\noindent\begin{tikzpicture}
\begin{semilogyaxis}[xlabel=t,ylabel=Точность решения,width=\textwidth, height=6cm]
\addplot[red, no markers] table [y=norma, x=t]{#1.txt};
\addplot[blue, no markers] table [y=normb, x=t]{#1.txt};
\addplot[black, no markers] table [y=norm, x=t]{#1.txt};
\end{semilogyaxis}
\end{tikzpicture}
}

\subsubsection{exp}

Функции: $u^s(x) = e^{0.5x}$, $u^c(x) = 5e^{x-3}$.

\graphgrid{3_grid_exp}

\noindent\textbf{Вывод:} аналогично предыдущей работе, наблюдается экстремум около точки $t=0.3$, но в данном случае этот экстремум не дает прироста точности больше, чем равномерная сетка.

\noindent\textbf{Вывод:} аналогично предыдущей работе, наблюдюается экстремум возле $t=1$, при котором точность чу лучше чем при равномерной сетке.

\noindent\textbf{Вывод:} нет особого отличия в точности от сгущения к одному или к другому концу сетки.

\subsubsection{-exp}

Функции: $u^s(x) = e^{-0.5x}$, $u^c(x) = 5e^{-x+3}$.

\graphgrid{3_grid_exp_minus}

\noindent\textbf{Вывод:} обращение предыдущей функции в обратную сторону никак не повлияло на точность.

\subsubsection{x2}

Функции: $u^s(x) = x^2$, $u^c(x) = 3x(x+2)-5$.

\graphgrid{3_grid_x2}

\noindent\textbf{Вывод:} предыдущие выводы не изменились.

\subsubsection{Сложная функция}

Функции: $u^s(x) = 3x^4 + 2e^x$, $u^c(x) = 6x-x^{e^x}$.

\graphgrid{3_grid_uuu}

\noindent\textbf{Вывод:} на этот раз в точке возле $t=0.3$ точность получилась больше, чем в равномерной сетке, а вот экстремум возле $t=1$ отсутствовал вовсе.

\section{Код программы}

\mycodeinput{c++}{../main.cpp}{main.cpp}