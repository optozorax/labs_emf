\pgfplotstableset{
	begin table={\rowcolors{2}{gray!25}{white}\begin{tabular}},
	end table=\end{tabular},
}

\mytitlepage{прикладной математики}{3}{Уравнения математической физики}{Решение нелинейных начально-краевых задач}{ПМ-63}{Шепрут И.И.}{1}{Патрушев И.И.}{2019}

\section{Цель работы}

Разработать программу решения гармонической задачи методом конечных элементов. Провести сравнение прямого и итерационного методов решения получаемой в результате конечноэлементной аппроксимации СЛАУ.

\section{Задание}

\begin{easylist}
\ListProperties(Hang=true, Margin2=12pt, Style2**=$\bullet$ , Hide2=2, Hide1=0)
& кмва
\end{easylist}

\textbf{Вариант 1:} Решить одномерную гармоническую задачу в декартовых координатах, базисные функции – линейные.

\section{Исследования}

кмва

\begin{easylist}
\ListProperties(Hang1=true, Margin2=12pt, Style1**=$\bullet$ , Hide1=1)
& кмва
\end{easylist}

\subsection{Изменение констант}

\newcommand{\bigtable}[1]{
\begin{center}
\noindent\pgfplotstabletypeset[
	columns/param/.style={string type, column name={Параметр}},
	columns/los_norm/.style={column name={LOS норма}},
	columns/bsg_norm/.style={column name={BSG норма}},
	columns/los_time/.style={column name={\tcell{LOS\\время}}},
	columns/bsg_time/.style={column name={\tcell{BSG\\время}}},
	columns/los_iter/.style={column name={\tcell{LOS\\итераций}}},
	columns/bsg_iter/.style={column name={\tcell{BSG\\итераций}}, column type/.add={}{|},},
	every head row/.style={before row=\hline,after row=\hline\hline}, 
	every last row/.style={after row=\hline},
	column type/.add={|}{},
	col sep=tab,
]{#1.txt}
\end{center}
}

\newcommand{\bigtables}[1]{
\bigtable{#1_omega}
\bigtable{#1_lambda}
\bigtable{#1_sigma}
\bigtable{#1_xi}
}

\bigtables{3_parameters_100}

\bigtables{3_parameters_50000}

\textbf{Вывод:}

\subsection{Зависимость точности от нелинейной сетки}

\newcommand{\graphgrid}[1]{
\noindent\begin{tikzpicture}
\begin{semilogyaxis}[xlabel=t,ylabel=Точность решения,width=\textwidth, height=6cm]
\addplot[red, no markers] table [y=norma, x=t]{#1.txt};
\addplot[blue, no markers] table [y=normb, x=t]{#1.txt};
\addplot[black, no markers] table [y=norm, x=t]{#1.txt};
\end{semilogyaxis}
\end{tikzpicture}
}

\graphgrid{3_grid_exp}

\graphgrid{3_grid_exp_minus}

\graphgrid{3_grid_x2}

\graphgrid{3_grid_uuu}

\section{Код программы}

% \mycodeinput{c++}{../main.cpp}{main.cpp}